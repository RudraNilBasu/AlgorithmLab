\documentclass[12pt]{article}

%\usepackage{fullpage}
%\usepackage{epic}
%\usepackage{eepic}
%\usepackage{graphicx}

\usepackage{listings} % Code
\usepackage{fancyhdr} % header footer
\usepackage{xcolor}  % Color
\usepackage{mathtools} % Math

\usepackage{geometry}
 \geometry{
 a4paper,
 total={170mm,257mm},
 left=20mm,
 top=20mm,
 }

%\newcommand{\proof}[1]{
%{\noindent {\it Proof.} {#1} \rule{2mm}{2mm} \vskip \belowdisplayskip}
%}


%\newtheorem{lemma}{Lemma}[section]
%\newtheorem{theorem}[lemma]{Theorem}
%\newtheorem{claim}[lemma]{Claim}
%\newtheorem{definition}[lemma]{Definition}
%\newtheorem{corollary}[lemma]{Corollary}

%\setlength{\oddsidemargin}{0in}
%\setlength{\topmargin}{0in}
%\setlength{\textwidth}{6.5in}
%\setlength{\textheight}{8.5in}

\cfoot{footer}

\lstset {
%language=Java,
backgroundcolor = \color{lightgray},
                   language = C++,
                   xleftmargin = 2mm,
                   framexleftmargin = 1em
%lineskip={-1.5pt}
}

%\usepackage[utf8]{inputenc}
 
 
% Information about contents section
\title{Contents}
\author{Rudra Nil Basu}
\date{ }
 
\renewcommand*\contentsname{Summary}

\begin{document}

% to generate the contents page
\maketitle
\tableofcontents

\newpage

\setlength{\fboxrule}{.5mm}\setlength{\fboxsep}{1.2mm}
\newlength{\boxlength}\setlength{\boxlength}{\textwidth}
\addtolength{\boxlength}{-4mm}
\begin{center}\framebox{\parbox{\boxlength}{\bf
CS593: Programming Practices using C++ \hfill 
Year: 2016
%Date: 11/10/2016
\\
%DATE
\hfill
}}\end{center}
\vspace{5mm}

\section{Pattern}

\textbf{Problem 1.1} \textit{Write a program to print the following pattern using \textbf{for} loop}\\

\begin{lstlisting}
1
22
333
4444
55555
...
\end{lstlisting}

\textit{Code.}

\begin{lstlisting}
#include<stdio.h>
using namespace std;

int main()
{
	int n,i,j;
	scanf("%d",&n);
	for(i=1;i<=n;i++) {
		for(j=1;j<=i;j++) {
			printf("%d",i);
		}
		printf("\n");
	}
}
\end{lstlisting}

\textit{Output.}
\begin{lstlisting}
5
1
22
333
4444
55555
\end{lstlisting}

\section{Average of Cricket Players}

\textbf{Problem 1.2} \textit{A cricket team has the following table of batting figures for a series of test matches
\setlength{\fboxrule}{.5mm}\setlength{\fboxsep}{1.2mm}
%\newlength{\boxlength}\setlength{\boxlength}{\textwidth}
\addtolength{\boxlength}{-4mm}
\begin{center}\framebox{\parbox{\boxlength}{\bf
Player's Name \hfill Runs \hfill Innings \hfill Times not out\\
Sachin \hfill 8430 \hfill 230 \hfill 180\\
Saurav \hfill 4200 \hfill 130 \hfill 9\\
Rahul \hfill 3350 \hfill 105 \hfill 11
}}\end{center}
\vspace{5mm}
% End of box
Write a program to read figures from the above form, to calculate the batting average and print out the complete table including the average
}
%\\


\textit{Code.}

\begin{lstlisting}
#include<stdio.h>
#include<vector>
#include<iostream>
using namespace std;

typedef struct stats {
	char name[50];
	int runs;
	int innings;
	int not_out;
	float average;
}stats;

int main()
{
	int i,n;
	char strtr[10];
	
	int ans;
	while(1) {
		stats players;
		printf("Enter name:");
		scanf("%s",players.name);
		printf("Enter runs, innings, not_out for %s\n",
		players.name);
		scanf("%d %d %d",&players.runs,&players.innings,
		&players.not_out);
		players.average=players.runs*1.0/players.innings;
		g.push_back(players);
		printf("Want more ?(1/0)\nyes=1\tno=0\n");
		scanf("%d",&ans);
		if(ans==0) {
			break;
		}
	}
	printf("Name\tRuns\tInnings\tNot Out\tAverage\n");
	for(i=0;i<g.size();i++) {
		printf("%s\t%d\t%d\t%d\t%f\n",g[i].name,
		g[i].runs,g[i].innings
				,g[i].not_out, g[i].average);
	}
	return 0;
}
\end{lstlisting}

\textit{Output.}
\begin{lstlisting}
Enter name:Rahul
Enter runs, innings, not_out for Rahul
3350 105 11
Want more ?(1/0)
yes=1   no=0
1
Enter name:Sachin
Enter runs, innings, not_out for Sachin
8430
230 18
Want more ?(1/0)
yes=1   no=0
1
Enter name:Saurav
Enter runs, innings, not_out for Saurav
4200 130 9
Want more ?(1/0)
yes=1   no=0
1
Enter name:ThePhenomenalRNB
Enter runs, innings, not_out for ThePhenomenalRNB
8888 105 18
Want more ?(1/0)
yes=1   no=0
0
Name    Runs    Innings Not Out Average
Rahul   3350    105     11      31.904762
Sachin  8430    230     18      36.652172
Saurav  4200    130     9       32.307693
ThePhenomenalRNB        8888    105     18      84.647621
\end{lstlisting}

\section{Electricity}

\textbf{Problem 1.3} \textit{Calculate electric charge for the following rates\\
For first 100 units \hfill 60P per unit\\
For next 200 units \hfill 80P per unit\\
Beyond 300 units \hfill 90P per unit\\
Minimum charge is Rs. 50.00. If total amount is more than 300.00, additional 15\% charge is added.\\
Read names of users and units consumed and print the charge with names
}\\

\begin{lstlisting}
1
22
333
4444
55555
...
\end{lstlisting}

\textit{Code.}

\begin{lstlisting}
#include<stdio.h>
using namespace std;

typedef struct charge {
	char name[50];
	int units;
	float cost;
}charge;

float findCost(int n)
{
	float c=0;
	if(n>=100) {
		c+=(100*0.6);
		n-=100;
	} else {
		c+=(n*0.6);
		return c;
	}
	if(n>=200) {
		c+=(200*0.8);
		n-=200;
	} else {
		c+=(n*0.8);
		return c;
	}
	if(n>0) {
		c+=(n*0.9);
		return c;
	}
}

int main()
{
	int n,i;
	scanf("%d",&n);
	charge chs[n];
	for(i=0;i<n;i++) {
		printf("Enter name:");
		scanf("%s",chs[i].name);
		printf("Enter no of units for %s\n",chs[i].name);
		scanf("%d",&chs[i].units);
		chs[i].cost=500.0;
		chs[i].cost+=findCost(chs[i].units);
		if(chs[i].cost>300) {
			chs[i].cost+=(0.15*chs[i].cost);
		}
	}
	for(i=0;i<n;i++) {
		printf("%s\t%d\t%f\n",chs[i].name,
		chs[i].units,chs[i].cost);
	}
	return 0;
}
\end{lstlisting}

\textit{Output.}
\begin{lstlisting}
3
Enter name:Rudra
Enter no of units for Rudra
250
Enter name:Tokon
Enter no of units for Tokon
10
Enter name:Rohit
Enter no of units for Rohit
300
Rudra   250     782.000000
Tokon   10      581.900024
Rohit   300     828.000000
\end{lstlisting}


\section{Election}

\textbf{Problem 1.4} \textit{An election is contested by five candidates, numbered 1-5.  Voting is done on ballot paper. Write a program to read the ballots and count the votes for each candidates. Any vote outside the range 1-5 is "split vote". Count the split votes as well
}\\

\textit{Code.}

\begin{lstlisting}
#include<stdio.h>
#include<string.h>
#include<algorithm>
#include<vector>
#include<queue>
#include<map>
#include<math.h>

#define ll long long int

int max(int a, int b)
{
	if(a>=b)
		return a;
	return b;
}

using namespace std;

int main()
{
	int n,i,count=0;
	int hash[7];
	memset(hash,0,sizeof(hash));
	scanf("%d",&n);
	while(n--) {
		count++;
		printf("Whom did %d vote for ? \n",count);
		int vote;
		scanf("%d",&vote);
		if(vote>=1 && vote<=5) {
			hash[vote]++;
		} else {
			hash[6]++;
		}
	}
	for(i=1;i<=5;i++) {
		printf("No of people voted for %d = %d\n",i,hash[i]);
	}
	printf("No of invalid votes = %d\n",hash[6]);
	return 0;
}
\end{lstlisting}

\textit{Output.}
\begin{lstlisting}
8
Whom did 1 vote for ? 
1
Whom did 2 vote for ? 
1
Whom did 3 vote for ? 
2
Whom did 4 vote for ? 
1
Whom did 5 vote for ? 
5
Whom did 6 vote for ? 
9
Whom did 7 vote for ? 
2
Whom did 8 vote for ? 
1
No of people voted for 1 = 4
No of people voted for 2 = 2
No of people voted for 3 = 0
No of people voted for 4 = 0
No of people voted for 5 = 1
No of invalid votes = 1
\end{lstlisting}

\section{Factorial}

\textbf{Problem 3.1} \textit{Calculate factorial of a number in C++ using functions}

\textit{Code.}

\begin{lstlisting}
#include<iostream>

using namespace std;

int fact(int n)
{
	if(n==1)
		return 1;
	return n*fact(n-1);
}

int main()
{
	int n;
	cout<<"Enter number"<<endl;
	cin>>n;
	cout<<"Factorial of "<<n<<": "<<fact(n)<<endl;
        return 0;
}

\end{lstlisting}

\textit{Output.}

\begin{lstlisting}
Enter number
5
Factorial of 5: 120
\end{lstlisting}

\section{Series sum}

\textbf{Problem 3.2} \textit{Calculate the sum of the series 1+22+32+42+... nth term in C++ using functions}

\textit{Code.}

\begin{lstlisting}
#include<iostream>

using namespace std;

int fact(int n)
{
	if(n==1)
		return 1;
	return n*fact(n-1);
}

int main()
{
	int n;
	cout<<"Enter number"<<endl;
	cin>>n;
	cout<<"Factorial of "<<n<<": "<<fact(n)<<endl;
        return 0;
}

\end{lstlisting}

\textit{Output.}

\begin{lstlisting}
Enter number
5
Factorial of 5: 120
\end{lstlisting}


\end{document}
\grid
