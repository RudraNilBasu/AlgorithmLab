\documentclass[12pt]{article}

%\usepackage{fullpage}
%\usepackage{epic}
%\usepackage{eepic}
%\usepackage{graphicx}

\usepackage{listings} % Code
\usepackage{fancyhdr} % header footer
\usepackage{xcolor}  % Color
\usepackage{mathtools} % Math

\usepackage{geometry}
 \geometry{
 a4paper,
 total={170mm,257mm},
 left=20mm,
 top=20mm,
 }

%\newcommand{\proof}[1]{
%{\noindent {\it Proof.} {#1} \rule{2mm}{2mm} \vskip \belowdisplayskip}
%}


%\newtheorem{lemma}{Lemma}[section]
%\newtheorem{theorem}[lemma]{Theorem}
%\newtheorem{claim}[lemma]{Claim}
%\newtheorem{definition}[lemma]{Definition}
%\newtheorem{corollary}[lemma]{Corollary}

%\setlength{\oddsidemargin}{0in}
%\setlength{\topmargin}{0in}
%\setlength{\textwidth}{6.5in}
%\setlength{\textheight}{8.5in}

\cfoot{footer}

\lstset {
%language=Java,
backgroundcolor = \color{lightgray},
                   language = Java,
                   xleftmargin = 2mm,
                   framexleftmargin = 1em
%lineskip={-1.5pt}
}

\begin{document}

\setlength{\fboxrule}{.5mm}\setlength{\fboxsep}{1.2mm}
\newlength{\boxlength}\setlength{\boxlength}{\textwidth}
\addtolength{\boxlength}{-4mm}
\begin{center}\framebox{\parbox{\boxlength}{\bf
CS504D: Object Oriented Programming \hfill 
Date: 
\\
%DATE
\hfill
}}\end{center}
\vspace{5mm}

\section{Week 1}

%The text of the notes goes here.
%To include a theorem with proof, use the following format.

\textbf{Problem 1.1} \textit{Write a program in Java to find the prime numbers between 1 to 100}\\


\textit{Code.}



\begin{lstlisting}[language=Java]
class prime
{
	static boolean primes[];
	public static void fillFalse()
	{
		int i;
		for(i=0;i<101;i++) {
			primes[i]=true;
		}
	}
	public static void initialise()
	{
		fillFalse();
		int i,j;
		primes[1]=false;
		for(i=2;i<101;i++) {
			if(primes[i]==true) {
				for(j=i+i;j<101;j+=i) {
					primes[j]=false;
				}
			}
		}
	}
	public static void print()
	{
		int i;
		for(i=1;i<=100;i++) {
			if(primes[i]==true)
			System.out.println(i);
		}
	}
	public static void main(String ags[])
	{
		primes=new boolean[101];
		initialise();
		print();
	}
}
\end{lstlisting}

\textbf{Problem 1.2} \textit{Write a program in Java to reverse a given number.}\\


\textit{Code.}

\begin{lstlisting}[language=Java]
import java.io.*;
class reverse
{
	public static void main(String args[])throws IOException
	{
		BufferedReader br=new BufferedReader(new 
		InputStreamReader(System.in));
		int n;
		n=Integer.parseInt(br.readLine());
		int m=n,rev=0;
		while(m>0) {
			rev=(rev*10)+m%10;
			m/=10;
		}
		System.out.println("Reversed Number "+rev);
	}
}
\end{lstlisting}

\textbf{Problem 1.3} \textit{Write a program in Java to find the sum of digits of a given number.}\\


\textit{Code.}

\begin{lstlisting}[language=Java]
import java.io.*;
class sum
{
	public static void main(String args[])throws IOException
	{
		BufferedReader br=new BufferedReader(new 
		InputStreamReader(System.in));
		int n;
		n=Integer.parseInt(br.readLine());
		int m=n,rev=0,sum=0;
		while(m>0) {
			sum+=(m%10);
			m/=10;
		}
		System.out.println("Sum of each digits "+sum);
	}
}
\end{lstlisting}

\textbf{Problem 1.4} \textit{Write a program in Java to print the following pattern.}

\begin{lstlisting}
*
**
***
****
\end{lstlisting}

\textit{Code.}

\begin{lstlisting}[language=Java]
class patt1
{
	public static void main(String args[])
	{
		int n=4,i,j;
		for(i=1;i<=n;i++) {
			for(j=1;j<=i;j++) {
				System.out.print("*");
			}
			System.out.println();
		}
	}
}
\end{lstlisting}

\textbf{Problem 1.5} \textit{Write a program in Java to print the following pattern.}

\begin{lstlisting}
   *
  ***
 *****
*******
\end{lstlisting}

\textit{Code.}

\begin{lstlisting}[language=Java]
class patt2
{
	public static void main(String args[])
	{
		int n=4,i,j,k;
		for(i=1;i<=n;i++) {
			for(j=n-1;j>=i;j--) {
				System.out.print(" ");
			}
			for(j=1;j<=((2*i)-1);j++) {
				System.out.print("*");
			}
			System.out.println();
		}
	}
}
\end{lstlisting}


\textbf{Problem 1.6} \textit{Write a program in Java to print the following pattern.}

\begin{lstlisting}
   *
  **
 ***
****
\end{lstlisting}

\textit{Code.}

\begin{lstlisting}[language=Java]
class patt3
{
	public static void main(String args[])
	{
		int n=4,i,j;
		for(i=1;i<=n;i++) {
			for(j=n-1;j>=i;j--) {
				System.out.print(" ");
			}
			for(j=1;j<=i;j++) {
				System.out.print("*");
			}
			System.out.println();
		}
	}
}
\end{lstlisting}

\newpage

% Week 2

\setlength{\fboxrule}{.5mm}\setlength{\fboxsep}{1.2mm}
%\newlength{\boxlength}\setlength{\boxlength}{\textwidth}
\addtolength{\boxlength}{-4mm}
\begin{center}\framebox{\parbox{\boxlength}{\bf
CS504D: Object Oriented Programming \hfill 
Date: 
\\
%DATE
\hfill
}}\end{center}
\vspace{5mm}

\section{Week 2 - Function and Constructor Overloading}

%Function and Constructor Overloading

\textbf{Problem 2.1} \textit{Write a program in Java to calculate the area of different shapes using function overloading.}


\textit{Code.}

\begin{lstlisting}[language=Java]
class area1
{
	void area(int sq)
	{
		System.out.println("Area of Square = "+sq*sq);
	}
	void area(int l, int w)
	{
		System.out.println("Area of Rectangle = "+l*w);
	}
	void area(float b,float ht)
	{
		System.out.println("Area of Triangle = "+(0.5)*(b*ht));
	}
	public static void main(String args[])
	{
		area1 a1=new area1();
		a1.area(10);
		a1.area(10,20);
		a1.area(10.0f,25.0f);
	}
}
\end{lstlisting}

\textit{Output.}


\textbf{Problem 2.1} \textit{Write a program in Java to calculate the area of different shapes using Constructor overloading.}


\textit{Code.}

\begin{lstlisting}[language=Java]
class area2
{
	area2(int sq)
	{
		System.out.println("Area of Square = "+sq*sq);
	}
	area2(int l, int w)
	{
		System.out.println("Area of Rectangle = "+l*w);
	}
	area2(float b,float ht)
	{
		System.out.println("Area of Triangle = "+(0.5)*(b*ht));
	}
	public static void main(String args[])
	{
		area2 a1=new area2(5);
		a1=new area2(12,20);
		a1=new area2(12.5f,13.0f);
	}
}
\end{lstlisting}

\textit{Output.}

\newpage


\section{Week 3 - }

\textbf{Problem 3.1} \textit{Write a program to design a class representing a bank account. The class should have the following data members:\\
* a/c no. * customer id * balance amount\\
The class should have member methods with the following functions:\\
* initialize initial value
* to deposit amount
* to withdraw amount
* to display customer id, a/c no. and current balance.}


\textit{Code.}

\begin{lstlisting}[language=Java]
import java.util.*;
class Bank{
	static Scanner sc=new Scanner(System.in);
	static long acno; static double amt;
	static String id;
	private void init(){
		acno=0; amt=0.0;
		id="";
	}
	private double deposit(double d){ return amt+=d; }
	private double withdraw(double d){
		if(d<amt&&amt!=0)return amt-=d;
		else {
			 System.out.println("Not Enough Balance!!");
			  return amt;
		}
	}
	private void print(){
		System.out.println("Customer ID \t A/c No. \t"+
		"Current Balance");
		System.out.println(id+"\t \t "+acno+"\t \t "+amt);
	}
	public static void main(String[]args){
		Bank obj=new Bank();
		obj.init();
		System.out.println("Enter account no and current balance:");
		id="3000114022";
		acno=sc.nextLong(); amt=sc.nextDouble();
		double d=0.0;
		int choice=0;
		do{
			System.out.println("Main Menu");
			System.out.println("0. Deposit");
			System.out.println("1. Withdrawal");
			System.out.println("2. Print Statement");
			System.out.println("3. Exit");
			System.out.println("Enter choice:");
			choice=sc.nextInt();
			switch(choice){
				case 0:d=0.0;
				        System.out.println("Enter "+
				        "amount to deposit:");
				        d=sc.nextDouble();
					 System.out.println("Deposit="+d+
					 "current balance="+
					 (double)obj.deposit(d));
					 break;
				case 1:d=0.0;
				       System.out.println("Amount ?");
				       d=sc.nextDouble();
				       System.out.println("withdrawal="+d+
				       "current balance="+
				       (double)obj.withdraw(d));
				        break;
				case 2:obj.print();
				       break;
				default:
					break;
			}
		}while(choice<3);
	}
}

\end{lstlisting}

\textit{Output.}


\textbf{Problem 3.2} \textit{Write a program to add two complex numbers.\\
Print the result in \(x+iy\)\\
form. Use objects as arguments to a method which will perform the addition
and use function overloading.}


\textit{Code.}

\begin{lstlisting}
class Complex
{
	double x;
	int y;
	Complex(double a,int b){
		x=a; y=b;
	}
	void print(){ System.out.println(x+"+ i"+y); }
}
class Test{
	double real; int imag;
	private double sum(double a,double b){
		real=a+b; return real;
	}
	private int sum(int a,int b){
		imag=a+b; return imag;
	}
	public static void main(String[]args){
		Complex obj=new Complex(4,6);
		obj.print();
		Complex obj1=new Complex(1,9);
		obj1.print();
		Test t1=new Test();
		System.out.println("sum =: "+t1.sum(obj.x,obj1.x)+"+"i
		+t1.sum(obj.y,obj1.y));
	}
}

\end{lstlisting}

\textit{Output.}

\newpage



%\begin{theorem}
%This is a theorem statement.
%\label{thm:sample-statement}
%\end{theorem}

%\proof{
%This is a proof.
%}



\end{document}
\grid
