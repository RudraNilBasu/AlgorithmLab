\documentclass[12pt]{article}

\usepackage{fullpage}
\usepackage{epic}
\usepackage{eepic}
\usepackage{graphicx}

\usepackage{listings} % Code
\usepackage{fancyhdr} % header footer

\newcommand{\proof}[1]{
{\noindent {\it Proof.} {#1} \rule{2mm}{2mm} \vskip \belowdisplayskip}
}


\newtheorem{lemma}{Lemma}[section]
\newtheorem{theorem}[lemma]{Theorem}
\newtheorem{claim}[lemma]{Claim}
\newtheorem{definition}[lemma]{Definition}
\newtheorem{corollary}[lemma]{Corollary}

\setlength{\oddsidemargin}{0in}
\setlength{\topmargin}{0in}
\setlength{\textwidth}{6.5in}
\setlength{\textheight}{8.5in}

\cfoot{footer}

\begin{document}

\setlength{\fboxrule}{.5mm}\setlength{\fboxsep}{1.2mm}
\newlength{\boxlength}\setlength{\boxlength}{\textwidth}
\addtolength{\boxlength}{-4mm}
\begin{center}\framebox{\parbox{\boxlength}{\bf
CS504D: Object Oriented Programming \hfill 
Date: 
\\
%DATE
\hfill
}}\end{center}
\vspace{5mm}

\section{Week 1}

%The text of the notes goes here.
%To include a theorem with proof, use the following format.

\textbf{Problem 1.1} \textit{Write a program in Java to find the prime numbers between 1 to 100}\\


\textit{Code.}

\begin{lstlisting}[language=Java]
class prime
{
	static boolean primes[];
	public static void fillFalse()
	{
		int i;
		for(i=0;i<101;i++) {
			primes[i]=true;
		}
	}
	public static void initialise()
	{
		fillFalse();
		int i,j;
		primes[1]=false;
		for(i=2;i<101;i++) {
			if(primes[i]==true) {
				for(j=i+i;j<101;j+=i) {
					primes[j]=false;
				}
			}
		}
	}
	public static void print()
	{
		int i;
		for(i=1;i<=100;i++) {
			if(primes[i]==true)
			System.out.println(i);
		}
	}
	public static void main(String ags[])
	{
		primes=new boolean[101];
		initialise();
		print();
	}
}
\end{lstlisting}

\textbf{Problem 1.2} \textit{Write a program in Java to reverse a given number.}\\


\textit{Code.}

\begin{lstlisting}[language=Java]
import java.io.*;
class reverse
{
	public static void main(String args[])throws IOException
	{
		BufferedReader br=new BufferedReader(new InputStreamReader(System.in));
		int n;
		n=Integer.parseInt(br.readLine());
		int m=n,rev=0;
		while(m>0) {
			rev=(rev*10)+m%10;
			m/=10;
		}
		System.out.println("Reversed Number "+rev);
	}
}
\end{lstlisting}

\textbf{Problem 1.3} \textit{Write a program in Java to find the sum of digits of a given number.}\\


\textit{Code.}

\begin{lstlisting}[language=Java]
import java.io.*;
class sum
{
	public static void main(String args[])throws IOException
	{
		BufferedReader br=new BufferedReader(new InputStreamReader(System.in));
		int n;
		n=Integer.parseInt(br.readLine());
		int m=n,rev=0,sum=0;
		while(m>0) {
			sum+=(m%10);
			m/=10;
		}
		System.out.println("Sum of each digits "+sum);
	}
}
\end{lstlisting}

\textbf{Problem 1.4} \textit{Write a program in Java to print the following pattern.}

\begin{lstlisting}
*
**
***
****
\end{lstlisting}

\textit{Code.}

\begin{lstlisting}[language=Java]
class patt1
{
	public static void main(String args[])
	{
		int n=4,i,j;
		for(i=1;i<=n;i++) {
			for(j=1;j<=i;j++) {
				System.out.print("*");
			}
			System.out.println();
		}
	}
}
\end{lstlisting}

\textbf{Problem 1.5} \textit{Write a program in Java to print the following pattern.}

\begin{lstlisting}
   *
  ***
 *****
*******
\end{lstlisting}

\textit{Code.}

\begin{lstlisting}[language=Java]
class patt2
{
	public static void main(String args[])
	{
		int n=4,i,j,k;
		for(i=1;i<=n;i++) {
			for(j=n-1;j>=i;j--) {
				System.out.print(" ");
			}
			for(j=1;j<=((2*i)-1);j++) {
				System.out.print("*");
			}
			System.out.println();
		}
	}
}
\end{lstlisting}


\textbf{Problem 1.6} \textit{Write a program in Java to print the following pattern.}

\begin{lstlisting}
   *
  **
 ***
****
\end{lstlisting}

\textit{Code.}

\begin{lstlisting}[language=Java]
class patt3
{
	public static void main(String args[])
	{
		int n=4,i,j;
		for(i=1;i<=n;i++) {
			for(j=n-1;j>=i;j--) {
				System.out.print(" ");
			}
			for(j=1;j<=i;j++) {
				System.out.print("*");
			}
			System.out.println();
		}
	}
}
\end{lstlisting}



%\begin{theorem}
%This is a theorem statement.
%\label{thm:sample-statement}
%\end{theorem}

%\proof{
%This is a proof.
%}



\end{document}
\grid
