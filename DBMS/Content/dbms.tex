\documentclass[12pt]{article}

%\usepackage{fullpage}
%\usepackage{epic}
%\usepackage{eepic}
%\usepackage{graphicx}

\usepackage{listings} % Code
\usepackage{fancyhdr} % header footer
\usepackage{xcolor}  % Color
\usepackage{mathtools} % Math
\usepackage{tabularx}
\usepackage{csvsimple, filecontents}

\usepackage{geometry}
 \geometry{
 a4paper,
 total={170mm,257mm},
 left=20mm,
 top=20mm,
 }

%\newcommand{\proof}[1]{
%{\noindent {\it Proof.} {#1} \rule{2mm}{2mm} \vskip \belowdisplayskip}
%}


%\newtheorem{lemma}{Lemma}[section]
%\newtheorem{theorem}[lemma]{Theorem}
%\newtheorem{claim}[lemma]{Claim}
%\newtheorem{definition}[lemma]{Definition}
%\newtheorem{corollary}[lemma]{Corollary}

%\setlength{\oddsidemargin}{0in}
%\setlength{\topmargin}{0in}
%\setlength{\textwidth}{6.5in}
%\setlength{\textheight}{8.5in}

\cfoot{footer}

\lstset {
%language=Java,
backgroundcolor = \color{lightgray},
                   language = C++,
                   xleftmargin = 2mm,
                   framexleftmargin = 1em
%lineskip={-1.5pt}
}

%\usepackage[utf8]{inputenc}
 
 
% Information about contents section
%\title{Contents}
%\author{Rudra Nil Basu}
\date{ }
 
%\renewcommand*\contentsname{Summary}

\begin{document}

% to generate the contents page
%\maketitle
\tableofcontents

\newpage

\setlength{\fboxrule}{.5mm}\setlength{\fboxsep}{1.2mm}
\newlength{\boxlength}\setlength{\boxlength}{\textwidth}
\addtolength{\boxlength}{-4mm}
\begin{center}\framebox{\parbox{\boxlength}{\bf
CS693: Object Oriented Programming Lab \hfill 
Year: 2017
%Date: 11/10/2016
\\
%DATE
\hfill
}}\end{center}
\vspace{5mm}

\section{Week 1: Players}

\subsection{Display player name, team id, stadium, and mdate for every German player}

\textit{Code.}

\begin{lstlisting}
select goal.player, goal.teamid, game.stadium, game.mdate
from goal join game on goal.matchid=game.id
where goal.teamid='GER'
\end{lstlisting}

\textit{Output.}\\
\csvautotabular{output/output1.csv}

\subsection{Show team1 and team2 and player for every goal scored by a player named 'Mario'}

\textit{Code.}

\begin{lstlisting}
select game.team1, game.team2, goal.player from goal
join game on goal.matchid=game.id where goal.player like 'Mario %'
\end{lstlisting}

\textit{Output.}\\
\csvautotabular{output/output2.csv}

\subsection{Show player, teamid, coach, gtime, for all goals scored in the first 10 mins}

\textit{Code.}

\begin{lstlisting}
select goal.player, goal.teamid, eteam.coach, goal.gtime
from goal join eteam on goal.teamid=eteam.id where goal.gtime <= 10
\end{lstlisting}

\textit{Output.}\\
\csvautotabular{output/output3.csv}

\subsection{List the player for every goal scored in a game whre the stadium was 'National Stadium, Warsaw'}

\textit{Code.}

\begin{lstlisting}
select goal.player
from game join goal
on game.id=goal.matchid
where game.stadium='National Stadium, Warsaw'
\end{lstlisting}

\textit{Output.}\\
\csvautotabular{output/output4.csv}

\subsection{Show the name of the player who scored a goal against Germany('GER')}

\textit{Code.}

\begin{lstlisting}
select goal.player
from game join goal
on goal.matchid=game.id
where (game.team2='GER' or game.team1='GER') and goal.teamid!='GER'
\end{lstlisting}

\textit{Output.}\\
\csvautotabular{output/output5.csv}

\subsection{Show the stadium and the number of goals scored in each stadium. Use count()}

\textit{Code.}

\begin{lstlisting}
select game.stadium, count(*)
from goal join game
on goal.matchid=game.id
group by game.stadium
order by count(*) asc
\end{lstlisting}

\textit{Output.}\\
\csvautotabular{output/output6.csv}

\subsection{List dates of all matches with 'Fernando Santos' as team1 coach}

\textit{Code.}

\begin{lstlisting}
select game.mdate, eteam.teamname
from game join eteam
on game.team1=eteam.id
where eteam.coach='Fernando Santos'
\end{lstlisting}

\textit{Output.}\\
\csvautotabular{output/output7.csv}

\section{Week 2: Joins}

\subsection{Insert a tupple in eteam}

\textit{Code.}

\begin{lstlisting}
insert into eteam (id, teamname, coach) values ('ARG', 'Argentina',
 'Diego Maradona')


1 row(s) inserted.

0.13 seconds
\end{lstlisting}

\subsection{Show the matchdate,stadium,player name,goal time for match ids 1019 and 1021}

\textit{Code.}

\begin{lstlisting}
select game.mdate, game.stadium, goal.player, goal.gtime
from game left join goal
on goal.matchid=game.id
where game.id=1019 or game.id=1021
\end{lstlisting}

\textit{Output.}\\
\csvautotabular{output/output8.csv}

\subsection{Show all the goal time,player, stadium where first team was Germany}

\textit{Code.}

\begin{lstlisting}
select goal.gtime, goal.player, game.stadium
from game left join goal
on goal.matchid=game.id
where game.team1='GER'
\end{lstlisting}

\textit{Output.}\\
\csvautotabular{output/output9.csv}

\subsection{Show the matchdate,stadium,teamname where Diego Maradona was the coach. Use right join}

\textit{Code.}

\begin{lstlisting}
select game.mdate, game.stadium, eteam.teamname
from game right join eteam
on game.team1=eteam.id or game.team2=eteam.id
where eteam.coach='Diego Maradona'
\end{lstlisting}

\textit{Output.}\\
\csvautotabular{output/output10.csv}

\subsection{Show id,matchdate,stadium,coach and team name for all team2. Use full outer join}

\textit{Code.}

\begin{lstlisting}
select game.id, game.mdate, game.stadium, eteam.coach, eteam.teamname
from game full outer join eteam
on game.team2=eteam.id
\end{lstlisting}

\textit{Output.}\\
\csvautotabular{output/output11.csv}

\section{Week 3: Employee}

\subsection{Show employee name and Manager name for Employee ID 4 and 6}

\textit{Code.}

\begin{lstlisting}
select b.name as Employee, a.name as Manager 
from employee a,employee b
where a.empid=b.manager and (b.empid=4 or b.empid=6)
\end{lstlisting}

\textit{Output.}\\

\subsection{Show the entire list of employee name and employee id under manager "Palash Roy"}

\textit{Code.}

\begin{lstlisting}
select distinct emp.name as Employee, man.name as Manager 
from employee emp,employee man
where emp.manager=man.empid and man.empid=4
\end{lstlisting}

\textit{Output.}\\

\section{Week 4: Client Master}

\subsection{Retrieve the list of names, city and the states of all the clients}

\textit{Code.}

\begin{lstlisting}
select name, city, state from client_master
\end{lstlisting}

\textit{Output.}\\

\subsection{List all the clients who are located in Mumbai}

\textit{Code.}

\begin{lstlisting}
select name from client_master
where city='Mumbai'
\end{lstlisting}

\textit{Output.}\\

\subsection{List the various products available from the PRODUCT-MASTER table}

\textit{Code.}

\begin{lstlisting}
select description from product_master
\end{lstlisting}

\textit{Output.}\\

\subsection{Change the city of Client\_No 'C00005' to Bangalore}

\textit{Code.}

\begin{lstlisting}
update client_master
set city='Bangalore'
where client_no='C00005'
\end{lstlisting}

\textit{Output.}\\

\subsection{Add a column called 'Telephone' of datatype 'number' and size='10' to the Client\_Master table}

\textit{Code.}

\begin{lstlisting}
alter table client_master
add Telephone number(10)
\end{lstlisting}

\textit{Output.}\\

\subsection{Delete the column 'Telephone'}

\textit{Code.}

\begin{lstlisting}
alter table client_master
drop column Telephone
\end{lstlisting}

\textit{Output.}\\

\subsection{Change the name of Client\_Master to Clnt\_Mast}

\textit{Code.}

\begin{lstlisting}
rename client_master to clnt_mast
\end{lstlisting}

\textit{Output.}\\

\subsection{List the names of all clients having 'a' letter in their names}

\textit{Code.}

\begin{lstlisting}
select name from clnt_mast
where name like '%a%'
\end{lstlisting}

\textit{Output.}\\

\subsection{List the names of the clients who stay in 'Bangalore' or 'Mangalore'}

\textit{Code.}

\begin{lstlisting}
select name from clnt_mast
where city like '_angalore'
\end{lstlisting}

\textit{Output.}\\

\subsection{List the names of the clients having 'h' as the second letter in their names}

\textit{Code.}

\begin{lstlisting}
select name from clnt_mast
where name like '_h%'
\end{lstlisting}

\textit{Output.}\\

\subsection{List all the information from the Sales\_Order table for orders placed in the month of June}

\textit{Code.}

\begin{lstlisting}
select * from sales_order
where order_date like '%JUN%'
\end{lstlisting}

\textit{Output.}\\

\subsection{List the information for Client\_No 'C00001' and 'C00002'}

\textit{Code.}

\begin{lstlisting}
select * from sales_order
where client_no='C00001' or client_no='C00002'
\end{lstlisting}

\textit{Output.}\\

\subsection{List products whose selling price is greater than 500 and less than 700}

\textit{Code.}

\begin{lstlisting}
select * from product_master
where sell_price > 500 and sell_price < 700
\end{lstlisting}

\textit{Output.}\\

\subsection{List the names, city and states of clients who are not in the state of 'Maharashtra'}

\textit{Code.}

\begin{lstlisting}
select name,city,state from clnt_mast
where state <> 'Maharashtra'
\end{lstlisting}

\textit{Output.}\\

\subsection{List the products whose selling price is more than 500. Calculate a new selling price as, original selling price * 15. Rename the new column in the output of the above query as New\_Price.}

\textit{Code.}

\begin{lstlisting}
select description, sell_price*15 as new_price 
from product_master
where sell_price > 500
\end{lstlisting}

\textit{Output.}\\

\subsection{Count the number of products having sell\_price less than or equal to 500.}

\textit{Code.}

\begin{lstlisting}
select count(sell_price) from product_master
where sell_price < = 500
\end{lstlisting}

\textit{Output.}\\

\subsection{List the order number and day on which clients placed their order.}

\textit{Code.}

\begin{lstlisting}
select name,order_no,order_date,to_char(to_date(order_date,
'dd/mm/yyyy'), 'DY') as day
from sales_order join clnt_mast on 
clnt_mast.client_no=sales_order.client_no
\end{lstlisting}

\textit{Output.}\\

\section{Week 6}

\subsection{Show the stadium and the number of goals scored in each stadium. For every match where 'GER' scored, show matchid, match date and the number of goals scored by 'GER'.}

\textit{Code.}

\begin{lstlisting}
select matchid,mdate,count(*)
from game join goal 
on goal.matchid=game.id
where teamid='GER'
group by mdate,matchid
\end{lstlisting}

\textit{Output.}\\

\end{document}
%\grid\grid
\grid
